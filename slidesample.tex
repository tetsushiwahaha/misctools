\documentclass[unicode,12pt,aspectratio=169]{beamer}
%\usepackage[orientation=landscape, size=a0, scale=1.4]{beamerposter}
% 10, 11, 12, 14, 17, 20 available

%%% slide design %%%
%\usetheme[progressbar=frametitle]{metropolis}
\usetheme[progressbar=frametitle]{focus}
%%%% for focus theme
\definecolor{main}{RGB}{22, 53, 123}
\definecolor{background}{RGB}{240, 247, 255}
\setbeamercolor{progress bar}{fg=red}
\setbeamercolor{block body}{bg=yellow!30!white}
%%%% for metropolis theme
\definecolor{mDarkBrown}{HTML}{000099}
\definecolor{mDarkTeal}{HTML}{000055}
\definecolor{mLightBrown}{HTML}{EB000B}
\definecolor{mLightGreen}{HTML}{14B03D}
%\definecolor{mLightBrown}{HTML}{EB811B}
%\definecolor{mDarkBrown}{HTML}{604c38}
%\definecolor{mDarkTeal}{HTML}{23373b}

%%%% Font setting %%%%
\usepackage{zxjatype}
%\usepackage[hiragino-pro]{zxjafont}
%\usepackage[morisawa-pr6n]{zxjafont}
%\usepackage[kozuka-pro]{zxjafont}
%\usepackage[hiragino-pron]{zxjafont}
%\usepackage[hiragino-pro]{zxjafont}
%\setCJKmainfont[Scale=0.95]{Meiryo}
\setCJKmainfont[Scale=0.95]{BIZ UDGothic}
\setsansfont[
  BoldFont={Fira Sans SemiBold},
  ItalicFont={Fira Sans Italic},
  BoldItalicFont={Fira Sans SemiBold Italic}
]{Fira Sans}

%%%% PACKAGES %%%%
\usepackage{bxdpx-beamer}	% for dvipdfmx
\usepackage{multicol}
\usepackage[euler-digits]{eulervm}
\usepackage{bm}
\usepackage{xcolor}
\usepackage{graphicx}
\graphicspath{{./images/}}
\usepackage{listings,jlisting} 
\lstset{language={C},
	basicstyle=\ttfamily\footnotesize,
	commentstyle=\textit, classoffset=1,
	frame=tRBl, framesep=5pt,
	numbers=left, stepnumber=1,
	numberstyle=\footnotesize, tabsize=2,
	escapechar=!
}

\usepackage{metalogo}
\usepackage{hyperref}
\hypersetup{ colorlinks=true, urlcolor=blue, }
%\usepackage{ascmac}
%\usepackage{slashbox}
%\usepackage{amsmath}
%\usepackage{bxascmac}


%%%%%%%%%%%%%%%%%%%%%%%%%%%%%%%%%%%%%%%%%%%%%%%%%%%%%%%%%%%%%%%%%%%%%%%%


\title{9月以降の研究室運営}
\subtitle{---予想不能な将来に対して行動計画を立てるのだ---}
\author[T. Ueta]{上田 哲史}
\institute{徳島大学 情報センター}
\date{\today}


\begin{document}

\begin{frame}
\maketitle
\end{frame}

\section{9月に入るので}

\begin{frame}{スケジュール合わせ}
\begin{itemize}
\item 金曜10:30〜のゼミは9月内は実施するが,後期では改めて各自のスケジュ
ールを合わせて,確実に全員が(オンラインで)集まる時間を決定.
\item ウエタは以下の授業を予定しているので,
これらのある曜日以外だとうれしい.\par

\centering
\begin{tabular}{ccc}
月曜日 & 10:25〜11:55 & 電子回路 \\
火曜日 & 8:40〜10:10 & 技術英語 \\
金曜日 & 14:35〜17:30 & 電気回路 \\
\end{tabular}
\end{itemize}
\end{frame}

\begin{frame}{報告会での提示資料}
\begin{itemize}
\item オンラインの説明になるので,説明スライドが必要.
\item \LaTeX を使う練習はしてほしいところだが…
\item 何をどこまで取り組んでいて,どこにひっかかってるのか,合理的な時間
内にウエタが必要な指示を出せるよう,できるだけ豊富な情報を提示せよ
\item \textbf{数行しかないような報告は今後禁止.}
何も提示するものがなければオリジ
ナルのポエムでも発表せよ
\item いちいちスクショなどを\LaTeX に貼り付けるのは面倒だろう.
プレゼンとして一本のPDFを作ることにこだわらなくていいので,
とにかくTeamsの画面共有として出して意義のあるデータを用意すること
\end{itemize}
\end{frame}

\begin{frame}{コード共有}
あんまし個々の人のコードは見たくない気もするが,ま,今年はオンラインで卒
研突破まで考えるので…
\begin{itemize}
\item Dropboxをインストールし,その中で開発
\item \texttt{git} コマンドを使えるようにする.ターゲットとなるプロジェクト
単位でディレクトリを掘って,そこで\texttt{git commit} できるようにする
\item github にてアカウントを取得.\texttt{git push origin master}で
同期できるよう整備する.
\item 全部オープンなのもアレなので,A6 B研内部でのみ共有できるようになら
ないか?
\end{itemize}
以上のことが何のことか分からなければ,天羽に聞く!
\end{frame}

\end{document}
