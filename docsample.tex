\documentclass[a4j]{jsarticle}

\usepackage{newtxtext,newtxmath}
\usepackage{ascmac}
\usepackage{setspace}
\usepackage{enumitem}
\usepackage{listings,jlisting}
\usepackage{bm}
\usepackage[dvipdfmx]{graphicx, xcolor}
\usepackage[top=25mm, bottom=20mm, left=20mm, right=20mm]{geometry}

%\renewcommand{\baselinestretch}{1.0}

\usepackage[dvipdfmx,
	bookmarksopen=true,  
    bookmarksopenlevel=3, bookmarks=true, 
    bookmarksnumbered=true,
    pdftitle={a guide},            
    pdftoolbar=true, pdfmenubar=true,
    pdfwindowui=true, pdffitwindow=false,
    pdfpagelayout=SinglePage,
    colorlinks=true, 
    linkcolor=blue,  %linkcolor=black, 
    urlcolor=blue,   %urlcolor=black,  
    citecolor=blue,   %citecolor=black,  
    bookmarkstype=toc
]{hyperref}

\usepackage{pxjahyper}

\lstset{%
  language={C},
  basicstyle={\ttfamily},%
  identifierstyle={\small},%
  commentstyle={\small\itshape},%
  keywordstyle={\small\bfseries},%
  ndkeywordstyle={\small},%
  stringstyle={\small\ttfamily},
  frame={tb},
  breaklines=true,
  columns=[l]{fullflexible},%
  numbers=left,%
  xrightmargin=0zw,%
  xleftmargin=3zw,%
  numberstyle={\scriptsize},%
  stepnumber=1,
  numbersep=1zw,%
  lineskip=-0.5ex%
}

\title{\textbf{タイトル}}
\author{著者名 \thanks{\texttt{mailaddr@tokushima-u.ac.jp}}}

\begin{document}
\maketitle

\begin{abstract}
概要を,必要ならばここに書く
\end{abstract}

% 長い内容になるようならば,目次を入れたらいいかも
%\tableofcontents

\section{最初のセクション}

コードリストはこんな感じ
\begin{lstlisting}
% whoami
tetsushi
% date
2021年 6月 8日 火曜日 14時19分56秒 JST
\end{lstlisting}

\section{次のセクション}

\appendix 

\section{ここは付録}

\begin{thebibliography}{9}
\bibitem{one}
% 論文の場合.
% 姓と名の間に半角スペース
著者 姓名, 著者 姓名,
``コンマは引用符の中に入れる,''
雑誌名,
Vol. xx, No. xx, pp. xx--xx, 月, 西暦. 
DOI:xxxxxx
%
% 著書の場合
% 本のタイトルは引用符ではくくらない
\bibitem{two}
姓 名,本のタイトル,出版社, 出版社の所在地,西暦.
%
\bibitem{three}
% URLだけ示す例 閲覧日を記述する.
\url{https://www.ipsj.or.jp/faq/chosakuken-faq.html}
(2020年6月24日閲覧)
%
\bibitem{oda}
小田 忠雄,``数学の常識・非常識---由緒正しい\TeX 入力法,''
数学通信,Vol. 4, No. 1, pp. 95--112, May 1999.
\url{http://www.math.tohoku.ac.jp/tmj/oda_tex.pdf}
%
\end{thebibliography}

% bibtex を使う場合
%\bibliographystyle{jplain}
%\bibliography{main}

\end{document}
%
